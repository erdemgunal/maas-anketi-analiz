This executive summary presents the key findings of the Software Sector Salary Survey 2025, analyzing data from \SampleSize{} software professionals. The study reveals critical information about salary determinants, gender disparities, and the impact of the technology stack in the software industry.

\subsection*{Key Findings}

\textbf{Salary Distribution and Determinants:} The analysis reveals that the location of the company is the strongest predictor of salary outcomes, with significant differences between geographic regions. Work arrangement (remote, office, hybrid) shows a medium effect size on compensation, while React usage alone does not significantly impact salary levels.

\textbf{Gender Pay Gap:} A statistically significant gender pay gap of approximately 16\% is observed in the sample, highlighting ongoing disparities in the Turkish software sector that require the attention of both companies and policy makers.

\textbf{Technology Stack Impact:} Rather than individual technologies, combinations of modern front-end and robust back-end competencies correlate with higher compensation. The analysis identifies specific technology stacks that demonstrate higher return on investment for professionals.

\subsection*{Critical Insights}

\begin{itemize}[leftmargin=*]
  \item \textbf{Location Premium:} Company location accounts for the largest variance in salary outcomes, with international companies offering significantly higher compensation than domestic firms.
  \item \textbf{Work Arrangement Benefits:} Remote and hybrid work arrangements show positive salary associations, suggesting that flexible work policies may contribute to higher compensation.
  \item \textbf{Stack Synergy:} Technology professionals benefit more from diversified skill sets rather than specialization in single technologies.
  \item \textbf{Career Progression:} Clear salary progression patterns exist across junior, mid-level, and senior positions, with substantial increases at each level.
\end{itemize}

\subsection*{Methodological Strength}

The study employs rigorous statistical methods including hypothesis testing, effect size calculations, and comprehensive data cleaning procedures. Results are based on a substantial sample size with robust statistical power, providing reliable insights for decision-making across the software industry ecosystem.
