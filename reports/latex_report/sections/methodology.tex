Data processing and analysis followed an a priori plan with rigorous quality controls.

\subsection{Data Cleaning and Feature Engineering}
We converted salary ranges into numerical midpoints, assigned values for open-ended ranges, normalized text fields, and engineered binary indicators via one-hot encoding for multi-select questions (e.g., programming languages, frameworks, tools, work type, location). Ordinal encoding was applied to career level. The final cleaned dataset contains no missing values.

\subsection{Statistical Analyses}
Primary hypothesis tests examined: (1) effect of React usage on salary (independent \textit{t}-test), (2) effect of work arrangement on salary (one-way ANOVA), (3) effect of company location on salary (one-way ANOVA), (4) gender pay gap (independent \textit{t}-test). We computed effect sizes (Cohen's \textit{d}, eta-squared) and 95\% confidence intervals.

Secondary analyses included: two-way ANOVA for interaction between company location and work arrangement; hour-of-day participation patterns and salary variability; technology stack comparisons and return-on-investment (ROI) ranking; and career progression patterns across levels.

\subsection{Visualization Standards}
All charts adhere to a unified visual standard: figures are generated at 12 by 8 inches, 300 DPI PNG output, with consistent typographic hierarchy and a Viridis color palette. These figures are optimized for direct inclusion into this LaTeX report with width settings between 0.7 and 0.9 of the text width.
